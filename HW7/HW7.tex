\documentclass[12pt]{article}

\usepackage{amsmath}
\usepackage{amssymb}
\usepackage{amsthm}

%\usepackage[textheight=8.5in, margin=1.6in]{geometry}
\usepackage[total={6in,8in}]{geometry}

%\usepackage{tikz}
%\usepackage{pgfplots}
%\tikzstyle{ball} = [circle,shading=ball, ball color=black,
% minimum size=1mm,inner sep=1.3pt]

%\usepackage{graphicx}

\usepackage{enumitem}
\usepackage{tikz}


% \setlist[enumerate]{itemsep=2pt,leftmargin=*,label={(\alph*)}}

%\usepackage[colorlinks=true,citecolor=black,linkcolor=black,urlcolor=blue]{hyperref}

\newcommand{\comment}[1]{%
 \text{\phantom{(#1)}} \tag{#1}
}
\newcommand{\N}{\mathbb{N}}
\newcommand{\Z}{\mathbb{Z}}
\newcommand{\Q}{\mathbb{Q}}
\newcommand{\R}{\mathbb{R}}
\newcommand{\C}{\mathbb{C}}
\newcommand{\ve}{\varepsilon}
\DeclareMathOperator{\tr}{tr}
\DeclareMathOperator{\Span}{\mathrm{Span}}
\DeclareMathOperator{\rk}{\mathrm{rank}}
\DeclareMathOperator{\trace}{\mathrm{tr}}
\DeclareMathOperator{\im}{\mathrm{im}}
\DeclareMathOperator{\Sym}{\mathrm{Sym}}
\DeclareMathOperator{\Hom}{\mathrm{Hom}}

\newcounter{problem}
\renewcommand \thesection {\theproblem}
\newenvironment{prob}[1][]
{
\refstepcounter{problem}\par\medskip
\def\myArg{#1}
 \ifx\myArg\empty
 \noindent \textsc{Problem~\theproblem.} \rmfamily
 \else
 \noindent \textsc{Problem~\theproblem\ (\textnormal{#1}).} \rmfamily
 \fi
}
{
\medskip

}

\parskip = 0.05in
\parindent = 0.0in

\pagestyle{empty}

\begin{document}

\usetikzlibrary{automata, positioning, arrows}
\tikzset{
 ->, % makes the edges directed
 >=stealth, % makes the arrow heads bold
 node distance=3cm, % specifies the minimum distance between two nodes. Change if n
 every state/.style={thick, fill=gray!10}, % sets the properties for each ’state’ n
 initial text=$ $, % sets the text that appears on the start arrow
 }

\centerline{MATH/CSCI 387 Homework 7}
\centerline{Sima Nerush}
\centerline{Discussed problems with Harrison and Riley}
\bigskip

\begin{prob}Show that any PSPACE-hard language is also NP-hard. (Remember that ``NP-hard'' requires the same thing as NP-completeness, except that the language does not have to be in NP. PSPACE-hard is defined similarly.)

For a language $L_1$ to be NP-Hard requires that for all $A$ in NP, $A \le _p L_1$.

For a language $L_2$ to be Pspace-hard requires that for all $A$ in Pspace, $A \le _p L_2$.

\textbf{Proof:} Let $L \in$ PSPACE-hard. This implies that any $A \in$ PSPACE is no harder than $L$. Therefore, any $B \in$ NP is also no harder than $L$. This shows any $B \in$ NP can be reduced to $L$. Thus, $L$ is also NP-hard.
\end{prob}

\pagebreak

\begin{prob} The game Gomoku is played by two players on an n × n board. The players alternate placing pieces, with one placing red and the other placing blue. (The pieces must be placed on open spaces.) The winner is the first player to achieve a line of 5 consecutive markers (in a row, column, or diagonal). A position consists of a description of what stones are on the board and whose turn it is. Let $GOMOKU$ be the set of positions from which red can force a win. Show that $GOMOKU \in PSPACE$.

Formulated as a language:

$GOMOKU = \{ p\ | \ p \text{ is a position where red can force a win} \}$

Define a PSPACE algorithm $M$ for $GOMOKU$:

\begin{enumerate}

    \item Get a position as an input.
    \item If it is a leaf and a winning position of Red, accept.
    \item If it is a leaf and a winning position for Blue, reject.
    \item If It is Red's turn:
    \begin{enumerate}
        \item[1.] Make a recursive call of $M$ on the position corresponding to all possible next turns.
        \item[2.] Erase the contents of the tape, storing only the results of the recursive calls.
        \item[3.] If one of them accepts, accept. Else, reject.
    \end{enumerate}
    \item If It is Blue's turn:
    \begin{enumerate}
        \item[1.] Make a recursive call of $M$ on the position corresponding to all possible next turns.
        \item[2.] Erase the contents of the tape, storing only the results of the recursive calls.
        \item[3.] If all of them accept, accept. Else, reject.
    \end{enumerate}


\end{enumerate}

\textbf{Analysis:}

Recursion can have at most $n^2$ depth, because at every turn it takes at most $n^2$ subsequent turns to fill up the board.

Each level of recursion adds $O(n^2)$ characters to the tape because it takes $O(n^2)$ symbols to represent a board.

It also takes $O(n^2)$ time to check whether a board position is a win. (This implies that the check can also be done in polynomial space).



\end{prob}

\pagebreak

\begin{prob} Show that $PSPACE$ is closed under the star operation.

Given a string and a polynomial time decider for $L$, we want to determine whether our string is in $L^*$.

Define an algorithm $M$ for $L^*$:

\begin{enumerate}
    \item Get a string as input:

    \item If $s \in L$, return true.
    \item for every possible splitting into parts $l$ and $r$ of s:
    \begin{enumerate}
        \item[1.] Recursively call $M$ on part $l$ of the string. If it accepts, erase the calculations on the tape and recursively call $M$ on part $r$ of the string. If it accepts, accept.
        \item[2.] Erase the calculations on the tape. 
        \item[3.] reject.

    \end{enumerate}
\end{enumerate}

\textbf{Analysis:}

Recursion can have at most $n$ depth, because there are n characters in the input string, and we cannot split beyond this point.

Each level of recursion adds some polynomial number of characters to the tape because we have a polynomial space decider for $L$.

\end{prob}

\end{document}
